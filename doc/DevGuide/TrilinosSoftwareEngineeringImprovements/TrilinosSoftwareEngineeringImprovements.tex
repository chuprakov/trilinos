\documentclass[pdf,ps2pdf,11pt]{SANDreport}
\usepackage{pslatex}
%Local stuff
\usepackage{graphicx}
\usepackage{latexsym}
%\usepackage{color}
\usepackage[all]{draftcopy}
%
% Command to print 1/2 in math mode real nice
%
\newcommand{\myonehalf}{{}^1 \!\!  /  \! {}_2}

%
% Command to print over/under left aligned in math mode
%
\newcommand{\myoverunderleft}[2]{ \begin{array}{l} #1 \\ \scriptstyle #2 \end{array} }

%
% Command to number equations 1.a, 1.b etc.
%
\newcounter{saveeqn}
\newcommand{\alpheqn}{\setcounter{saveeqn}{\value{equation}}
\stepcounter{saveeqn}\setcounter{equation}{0}
\renewcommand{\theequation}
	{\mbox{\arabic{saveeqn}.\alph{equation}}}}
\newcommand{\reseteqn}{\setcounter{equation}{\value{saveeqn}}
\renewcommand{\theequation}{\arabic{equation}}}

%
% Shorthand macros for setting up a matrix or vector
%
\newcommand{\bmat}[1]{\left[ \begin{array}{#1}}
\newcommand{\emat}{\end{array} \right]}

%
% Command for a good looking \Re in math enviornment
%
\newcommand{\RE}{\mbox{\textbf{I}}\hspace{-0.6ex}\mbox{\textbf{R}}}
\newcommand{\CPLX}{\mbox{\textbf{C}}\hspace{-1.1ex}\mbox{l}\hspace{0.4ex}}

%
% Commands for Jacobians
%
\newcommand{\Jac}[2]{\displaystyle{\frac{\partial #1}{\partial #2}}}
\newcommand{\jac}[2]{\partial #1 / \partial #2}

%
% Commands for Hessians
%
\newcommand{\Hess}[2]{\displaystyle{\frac{\partial^2 #1}{\partial #2^2}}}
\newcommand{\hess}[2]{\partial^2 #1 / \partial #2^2}
\newcommand{\HessTwo}[3]{\displaystyle{\frac{\partial^2 #1}{\partial #2 \partial #3}}}
\newcommand{\hessTwo}[3]{\partial^2 #1 / (\partial #2 \partial #3)}



%\newcommand{\Hess2}[3]{\displaystyle{\frac{\partial^2 #1}{\partial #2 \partial #3}}}
%\newcommand{\myHess2}{\frac{a}{b}}
%\newcommand{\hess2}[3]{\partial^2 #1 / (\partial #2 \partial #3)}

%
% Shorthand macros for setting up a single tab indent
%
\newcommand{\bifthen}{\begin{tabbing} xxxx\=xxxx\=xxxx\=xxxx\=xxxx\=xxxx\= \kill}
\newcommand{\eifthen}{\end{tabbing}}

%
% Shorthand for inserting four spaces
%
\newcommand{\tb}{\hspace{4ex}}

%
% Commands for beginning and ending single spacing
%
\newcommand{\bsinglespace}{\renewcommand{\baselinestretch}{1.2}\small\normalsize}
\newcommand{\esinglespace}{}



\raggedright

% If you want to relax some of the SAND98-0730 requirements, use the "relax"
% option. It adds spaces and boldface in the table of contents, and does not
% force the page layout sizes.
% e.g. \documentclass[relax,12pt]{SANDreport}
%
% You can also use the "strict" option, which applies even more of the
% SAND98-0730 guidelines. It gets rid of section numbers which are often
% useful; e.g. \documentclass[strict]{SANDreport}

% ---------------------------------------------------------------------------- %
%
% Set the title, author, and date
%

\title{\center
Trilinos Software Engineering Improvement Ideas}
\author{
Roscoe A. Bartlett \\ Department of Optimization and Uncertainty Estimation \\ \\
Jim M. Willenbring \\ Department of Scalable Algorithms \\ \\
Sandia National Laboratories, Albuquerque NM 87185-1318 USA \\ }

% ---------------------------------------------------------------------------- %
% Set some things we need for SAND reports. These are mandatory
%
\SANDnum{SAND2008-xxx}
\SANDprintDate{October 2008}
\SANDauthor{Roscoe A. Bartlett and Jim M. Willenbring}

% ---------------------------------------------------------------------------- %
% The following definitions are optional. The values shown are the default
% ones provided by SANDreport.cls
%
\SANDreleaseType{Unlimited Release}
%\SANDreleaseType{Not approved for general release}

% ---------------------------------------------------------------------------- %
% The following definition does not have a default value and will not
% print anything, if not defined
%
%\SANDsupersed{SAND1901-0001}{January 1901}

% ---------------------------------------------------------------------------- %
%
% Start the document
%
\begin{document}

\maketitle

%% ------------------------------------------------------------------------ %
%% An Abstract is required for SAND reports
%%
%
%%\clearpage
%
%%
%\begin{abstract}
%%
%
%Blah blah blah ...
%
%%
%\end{abstract}
%%

%% ------------------------------------------------------------------------ %
%% An Acknowledgement section is optional but important, if someone made
%% contributions or helped beyond the normal part of a work assignment.
%% Use \section* since we don't want it in the table of context
%%
%\clearpage
%\section*{Acknowledgments}
%
%Blah blah blah ...

%
%The format of this report is based on information found
%in~\cite{Sand98-0730}.

% ------------------------------------------------------------------------ %
% The table of contents and list of figures and tables
% Comment out \listoffigures and \listoftables if there are no
% figures or tables. Make sure this starts on an odd numbered page
%
\clearpage
\tableofcontents
%\listoffigures
%\listoftables

% ---------------------------------------------------------------------- %
% An optional preface or Foreword
%\clearpage
%\section{Preface}
%Although muggles usually have only limited experience with
%magic, and many even dispute its existence, it is worthwhile
%to be open minded and explore the possibilities.

% ---------------------------------------------------------------------- %
% An optional executive summary

%\clearpage

%\section{Executive Summary}

% ---------------------------------------------------------------------- %
% An optional glossary. We don't want it to be numbered
%\clearpage
%\section*{Nomenclature}
%\addcontentsline{toc}{section}{Nomenclature}
%\begin{itemize}
%\item[alohomora]
%spell to open locked doors and containers
%\end{itemize}

% ---------------------------------------------------------------------- %
% This is where the body of the report begins; usually with an Introduction
%


\SANDmain % Start the main part of the report


%
{}\section{Introduction}
%

Here we outline a series of issues and suggestions for improving the
software engineering structure and practices of Trilinos to improve
day-to-day development and releases of Trilinos.


%
{}\section{Goals of Principles and Practices}
%

The goal of these principles and practices are to:

\begin{itemize}

{}\item Facilitate the rapid development of deep stacks of vertically
integrated Trilinos packages within a product application code using
all development sources

{}\item Improve the overall speed and productivity of Trilinos
development

{}\item Improve the overall release process to allow for faster, lower
risk, more predictable releases of Trilinos taken off of the main
development branch with greater frequency

{}\item Still allow the Trilinos software engineering infrastructure
to support very informal (from a software perspective) algorithmic
research efforts with low overhead and low impact on other Trilinos
development

{}\item Do all of this while still allowing Trilinos growth to
continue in a scalable way

\end{itemize}


%
{}\section{General Lean/Agile Software Engineering Principles}
%

\begin{itemize}
 
{}\item High quality software is developed in short fixed-time
iterations.

{}\item Software requirements should be constantly re-evaluated and
re-prioritized between iterations driven by real (customer) needs.

{}\item Software should be delivered to real (or as real as we can
make them) customers is short intervals.

{}\item The best software solutions are found through research and
experimentation and not primarily through ``planning ahead''
(i.e. see the �Wicked Problem'' in �Code Complete: 2nd Edition'').

{}\item High quality software is developed in small increments and
tested thoroughly between changes.

{}\item High quality defect-free software is most effectively
developed by not putting defects into the software in the first place
(i.e. code reviews, pair programming etc.).

{}\item High quality software is proceeded by and protected by very
though, automated tests (i.e. TDD).

{}\item Critical software pieces that must work together in a final
release should be integrated and tested together early and often
(i.e. daily and continuous integration).

{}\item Working software should never be allowed to be broken for any
period of time (i.e. don't introduce defects in working software or
in new software).  The idea that we develop shaky software and then
``harden'' it is not consistent with Lean/Agile best practices.

{}\item Pay down �technical debt'' when it accumulates and employ
constant refactoring to pay down this �debt'' and to avoid its
accumulation.  This avoids the common increase of software �entropy''.

{}\item Most mistakes that people make are due to a faulty
process/system (W. Edwards Deming).

{}\item Automation is needed to avoid mistakes and improve software
quality.

{}\item Avoid points of synchronization.  Allow people to work as
independently as possible and have the system set up to automatically
support this.

\end{itemize}


%
{}\section{Key Concepts and Defintions}
%


%
{}\subsection{``Primary'' verses ``Secondary'' Platforms}
%

\begin{itemize}

{}\item\textit{``Primary'' Platforms}: These are platforms where
developers do their day-to-day development work.  It is important to
identify these platforms because a certain set of issues must be
addressed for the primary development platforms.  Possible candidates
for ``Primary'' development platforms for Trilinos include:

  \begin{itemize}
  {}\item Linux + gcc 4.x (high warning levels and warnings as errors)
  {}\item Mac OSS + gcc 4.x (lesser warning levels)
  \end{itemize}

{}\item\textit{``Secondary'' Platforms}: These are platforms where
developers do not typically do their day-to-day development work but
instead primarily represent platforms that must be supported in the
release of the software.  Because day-to-day development is not
typically done on these platforms, a different set of issues arise and
different practices should be considered as compared to those for the
``Primary'' development platforms.  Examples of ``Secondary''
platforms include:

  \begin{itemize}
  {}\item Intel compilers on Linux
  {}\item PGI compilers on Linux
  {}\item IBM compilers on AIX
  {}\item Pathscale compilers on Linux
  {}\item SUN compilers on IRIS
  {}\item ...
  \end{itemize}

\end{itemize}


%
{}\subsection{``Stable'' verses ``Experimental'' Code and Tests}
%


%
{}\subsubsection{``Stable'' Code and Tests}
%

\begin{itemize}

{}\item\textit{``Stable'' Code} meets one or more of the following
criteria:

  \begin{itemize}

  {}\item Represents an important capability being used by an existing
  customer in a release of Trilinos, or

  {}\item Represents a new capability that the authors are willing to
  stand behind (as defined below) and is being targeted for the next
  release

  \end{itemize}

{}\item Sub-categorizations of ``Stable'' code based on
dependencies:

  \begin{itemize}

  {}\item\textit{``Primary Stable'' Code} is ``Stable'' code that only
  depends on:
  
    \begin{itemize}
    {}\item C, and C++ compilers
    {}\item Fortran 77 compiler
    {}\item BLAS and LAPACK
    {}\item MPI
    \end{itemize}

  {}\item\textit{``Secondary Stable'' Code} is ``Stable'' code with
  additional dependencies such as:

    \begin{itemize}
    {}\item SWIG/Python (i.e. PyTrilinos)
    {}\item Fortran 2003+ (i.e. ForTrilinos)
    {}\item External direct sparse solvers like UMFPACK, SuperLU,
    etc. (i.e. Amesos adapters)
    {}\item ...
    \end{itemize}

  \end{itemize}

{}\item ``Stable'' code/tests are expected to be kept working at all
times.  There should be little excuse for breaking ``Stable'' code on
the primary development platform(s).

{}\item ``Stable'' code in one package can only depend on ``Stable''
code in other packages.

{}\item ``Stable'' code should by default only build ``Primary
Stable'' code.  Enabling ``Secondary Stable'' code should require
explicit configure-time options.

{}\item ``Stable'' code should be developed from the start and
maintained to be highly portable.

{}\item ``Stable'' code should be maintained at the highest quality as
defined by Lean/Agile software engineering principles.

\end{itemize}


%
{}\subsubsection{``Experimental'' Code and Tests}
%

\begin{itemize}

{}\item By definition, {}\textit{`Experimental'' Code} is all
remaining code that is not ``Stable'' code.

{}\item ``Experimental'' code may represent fundamental research and
may be developed with informal low-quality software practices.

{}\item Any code that has a direct and mandatory dependency on any
``Experimental'' code must also be considered to be ``Experimental''
code.

{}\item Developers of one collection of ``Experimental'' code should
try to avoid depending on other ``Experimental'' code because it is
likely to be unstable and break frequently.  One exception to this
is in close research collaborations that seek to build new
vertically integrated capabilities.

{}\item ``Experimental'' code should be protected behind ifdefs with
macos that must be defined in order to be built.  All
``Experimental'' code and tests are disabled by default and must be
explicitly enabled.

\end{itemize}


%
{}\subsection{Synchronous verses Asynchronous Continuous Integration}
%


%
{}\subsubsection{Synchronous Continuous Integration}
%

{}\textit{``Synchronous Continuous Integration''} is an engineering
process where software is integrated and tested locally *before* it is
checked in by performing the following in rapid succession:

\begin{itemize}

{}\item Do a VC update to get all current changes in the version
control repository

{}\item Rebuild all affected code and rerun the ``precheckin'' test
suite (whatever we define that to be).  If there are *any* failing
tests, fix the code, investigate why the code fails, or do something
else to make sure it is okay to check in.  Don't just check in
broken code!

{}\item If all the affected code and tests build and pass, quickly
check in the changes.

\end{itemize}


%
{}\subsubsection{Asynchronous Continuous Integration}
%

{}\textit{``Asynchronous Continuous Integration''} is and engineering
process where woftware is integrated and tested on a CI server
{}\underline{after} it is checked in by performing the following:

\begin{itemize}

{}\item Developers do a reasonable amount of testing for changes in
code they have made but don't bring in all changes from the
version control repository and don't run all of the required
``integration'' tests.

{}\item Developers check in code without going through all of the
steps of full ``synchronous continuous integration''.

{}\item A continuous integration (CI) server periodically checks out
the code (on a fixed schedule or when changes are detected) and does
a full integration build and runs the integration test suite.

{}\item If the build or any tests fail, an email notification is
sent out to some group of people alerting them to the problem.
Ideally, everyone who checked in code since the last successful CI
build should get an email.

\end{itemize}


%
{}\section{Goals for Partioning of Code, Platforms, and Continuous
Integration Testing}
%


%
{}\subsection{Goals for Partitioning into ``Primary'' and
``Secondary'' Platforms}
%

The goal of formally partitioning platforms into ``Primary''
development platforms and other remaining ``Secondary'' platforms is
to allow us to define a different set of practices and expectations
for or code for each.  For example, since day-to-day development
occurs on ``Primary'' development platforms, we need to ensure that
the code always builds and all of the tests run in order to facilite
fast development cycles with frequent checkins.  Also, we need to
ensure that developers are writing and checking code that is very
portable so we want to use high warning levels and elevate warnings to
errors (see {}\cite[Item 1]{C++CodingStandards05}).  However, because
day-to-day development is not conducted on the ``Secondary''
platforms, we can legitimately reduce our intolerance a little for
broken builds and failing tests.


%
{}\subsection{Goals for Partitioning into ``Stable'' and
``Experimental'' Code and Tests}
%

\begin{itemize}

{}\item By explicitly partitioning off ``Experimental'' code, we allow
a great deal of crazy and impulsive algorithms research to be
conducted within Trilinos, have that work benefit from ready to use
``Stable'' building blocks (represented by the other ``Stable'' code),
and take advantage of everything the Trilinos environment has to
offer.

{}\item By maintaining a ``Stable'' core of code developed with a high
degree of software engineering discipline, other ``Experimental''
research efforts can remain highly productive because their foundation
is not constantly breaking.  At the same time, new requirements from
``Stable'' code needed to drive ``Experimental'' research code
development can be rapidly developed and integrated in real time.
This is *not* possible if ``Experimental'' efforts are based on a
static release of Trilinos.

{}\item By specifically partitioning off ``Experimental'' code from
``Stable'' code, we avoid the problem of a top-heavy overly strict
environment that is only geared toward immediate production needs and
does not allow for rapid research investigations.

{}\item By keeping ``Stable'' code in a near releasable state, we
allow for fast and frequent releases of Trilinos.

{}\item Summary: We can have our cake and eat it too!

\end{itemize}


%
{}\section{General Practices Related to ``Stable'' and
``Experimental'' Code}
%


%
{}\subsection{General practices related to ``Stable'' code}

\begin{itemize}

{}\item All ``Stable'' code should be developed and maintained with
stability and portability as the highest goals.

{}\item All ``Stable'' code should be maintained in a ``done''
(i.e. close-to-releasable) state with up-to-date tests, examples,
documentation (which are critical for allowing for fast \& frequent
releases).

{}\item All ``Stable'' code should be compiled with high warning
levels and treat warnings as errors (this improves the portability
of the software).

{}\item Backward compatibility for ``Stable'' code (for at least one
major release or more) is a high priority (this is critical to
successfully developing capabilities against production application
codes).

{}\item It is every Trilinos developer's responsibility to help
maintain the stability and integrity of ``Stable'' code.

{}\item All ``Stable'' code should be built every night on a variety
of different platforms \& compilers and developers should place a
high priority on fixing broken builds first and then on fixing
broken tests.

{}\item We must have a 100\% passing test policy for all ``Stable''
code on our primary development platforms.

{}\item We endeavor to have a 100\% passing test goal for all
``Stable'' code on other Secondary nightly testing platforms as
well.  However, we recognize that it is unrealistic and prohibitive
to ``stop the line'' for failures on Secondary test platforms.
Therefore, efforts to fix failing builds and tests on Secondary
platforms will take place in an Secondary development loop that runs
behind the efforts on the primary development platforms.  NOTE:
Requiring all ``Stable'' code to build and run with high warning
levels and warnings as errors massively improves the portability of
C/C++ code so we expect few problems porting to these Secondary
platforms.  Examples of secondary platforms include:

\end{itemize}


%
{}\subsection{General practices related to ``Experimental'' code}
%

\begin{itemize}

{}\item While other Trilinos developers have no direct
responsibility to maintain ``Experimental'' code, they should still
consider the impact their changes will have *before* they check in
and they should make the most basic attempts to keep
``Experimental'' code working if they know that they will break
something and not maintain backward compatibility.

Example: If you are changing the name of a ``Stable'' class, use a
script to automatically change the name in all source files in all
dependent packages, including ``Experimental'' code.  Don't just
``lean on the compiler'' when making changes.

{}\item The code and build system should be set up so that all
``Experimental'' code segments and even entire files can be removed
by an automated set of scripts.  We must test that after we have
removed ``Experimental'' code that we can still build the ``Stable''
code.  We must do this type of ``Experimental'' code removal
followed by testing of ``Stable'' code on a routine automated basis.

{}\item At least one central testing platform should be available
with *all* of the dependencies required to build all ``Secondary
Stable'' code that all Trilinos developers can access and build on
during normal business hours.  This is needed to allow detailed
testing of changes that might affect ``Secondary Stable'' code
either before a checkin is performed or after a checkin has been
performed and the ``asynchronous continuous integration'' test fails
and a developer needs to quickly diagnose and fix the problem.

\end{itemize}


%
{}\subsection{Promotion of ``Experimental'' code to ``Stable'' code}
%

TBD


%
{}\section{Continuous integration Testing of ``Stable'' code}
%


%
{}\subsection{Synchronous continuous integration for ``Primary
Stable'' code}
%

Use ``synchronous continuous integration'' as the defacto standard for
testing ``Primary Stable'' code before every checkin by applying the
following process:

\begin{itemize}

{}\item A) Start filling out the checkin checklist message in a
temporary text file (i.e. in a text file {}\texttt{checkin\_message}) copied
from the official Trilinos checkin checklist message template.

{}\item B) Do a VC update to get all current changes in the
version control repository

{}\item C) Configure Trilinos to enable all ``Primary Stable''
code that depends on your code:

  \begin{itemize}

  {}\item Exception: If the developer is very confident that the
  changes they are making will not break other ``Stable'' code,
  then they can chose not to enable and test all dependant
  ``Stable'' code.  However, when in doubt, enable *all* dependent
  code and tests!

  {}\item Configurations that should be tested:

    \begin{itemize}
    {}\item Without any ``Secondary Stable'' or ``Experimental''
    code enabled
    {}\item With and without MPI enabled
    {}\item With and without debug checking
    {}\item NOTE: testing serial + debug and then mpi + optimized
    should be sufficient
    \end{itemize}

  \end{itemize}

{}\item D) Rebuild and rerun the ``pre-checkin'' test suite
(whatever we define that to be).  If there are *any* tests fail,
fix the code, investigate why the tests are failing (Are the
failures your fault or not?  Are your changes are related or not?) 
to make sure it is okay to check in.  Don't just check in broken
tests.  DO NOT UNDER ANY CIRCUMSTANCE EVER CHECK IN CODE THAT DOES
NOT BUILD!

{}\item E) While the rebuild and retest is going on, finish
filling out the checkin checklist message
(i.e. {}\texttt{checkin\_message}).

{}\item F) If the rebuild/restest all passes and you are sure you
have not created any regressions (i.e. all tests pass, 0 test
fail), then:

  \begin{itemize}

  {}\item Quickly do a {}\texttt{cvs nq update dP} to see if there
  are any new changes checked into the CVS repository that might
  conflict with your changes.  If you see changes that are
  worrisome, go back to step {}\texttt{B} and repeat.

  {}\item Otherwise, if there are no new updates or if you are
  pretty sure that the updates do not conflict with your changes,
  then go ahead and check in the code in one global atomic checkin
  using the checkin checklist temporary text file you have been
  filling out with {}\texttt{cvs commit F
  {}\texttt{checkin\_message}}.

  NOTE: Do not check in your changes is different sets because you
  can (and will) make the VC repository inconsistent during your
  checkins.

  \end{itemize}

{}\item G) Otherwise, abort the checkin and then do either:

  \begin{itemize}

  {}\item Backup your changes to keep them safe (e.g. using
  {}\texttt{tar czvf Trilinos.date.tar.gz Trilinos} or something
  and {}\texttt{scp} the file to another machine).

  {}\item Or, Try to resolve the problems and get the code to
  build and get all of the tests to pass.

  \end{itemize}

{}\item NOTE \#1: Any and all of the above steps can be bypassed by
the developer based on their judgment but the safest approach is
to do them all.

{}\item NOTE \#2: A tool (or set of tools) must be available to
perform all of the above steps in order to help avoid human error.
This can be built into the Cmake build system run as a single
command.

A tool that will automatically enable all affected ``Primary
Stable'' code for a given set of changes will remove the
guess-work from the ``synchronous continuous integration'' testing
process => This is already supported in the current Cmake build
system for Trilinos!

{}\item NOTE \#3: The code and tests suites many need to be cleaned
up to allow for faster rebuilds and tests to make this feasible
(see the books ``Implementing Lean Software Development'' and
``Continuous Integration'')

{}\item NOTE \#4: The fallback to doing complete ``synchronous
continuous integration'' is ``synchronous continuous integration''
(see below).

{}\item NOTE \#5: The big advantage of ``synchronous continuous
integration'' is that any changes that break down-stream ``Primary
Stable'' code will be immediately obvious and caught *before* a
checkin.

\end{itemize}


%
{}\subsection{Asynchronous continuous integration for ``Secondary Stable'' code}
%

Use ``Asynchronous continuous integration'' as the primary means to
test ``Secondary Stable'' code while following the ``synchronous
continuous integration'' procedure should be used for any ``Primary
Stable'' code involved:

\begin{itemize}

{}\item Only ``Secondary Stable'' code that is being is locally
modified needs to be enabled and tested prior to checkin.

{}\item Use the process of ``synchronous continuous integration''
described above to protect all ``Primary Stable'' code.

{}\item Trilinos must be configured and tested for both of the
following separate configurations:

  \begin{itemize}

  {}\item With the ``Secondary Stable'' code being modified
  enabled along with the rest of the ``Primary Stable'' code
  
  {}\item Only ``Primary Stable'' code without the ``Secondary
  Stable'' code enabled

  \end{itemize}

{}\item The developer checks in the code according to the criteria
for ``synchronous continuous integration'' described above only
strictly followed w.r.t. to ``Primary Stable'' code.

{}\item One or more Continuous Integration (CI) Servers checkout
the code and build all ``Primary and Secondary Stable'' code and
send email alerts if any build or test failures are detected.

{}\item NOTE \#1: All Trilinos developers should get an email when
any aspect of the ``asynchronous continuous integration''
build/tests fail.

{}\item NOTE \#2: Before getting updated source from the version
control repository, developers should check the ``asynchronous
continuous integration'' results dashboard to make sure that the
build is not broken.  A stable recent version of the code can be
obtained by using the date stamp for a recent successful
continuous integration build:

{}\item NOTE \#3: The big disadvantage of ``asynchronous continuous
integration'' is that it will be hard for a people who break
down-stream ``Primary Stable'' code to realize that they have and
get it reported correctly.

\end{itemize}


%
{}\section{Daily integration testing of ``Stable'' and
``Experimental'' code}
%


%
{}\subsection{Nightly testing and portability testing of ``Stable''
code}
%

\begin{itemize}

{}\item In nightly testing, we will test *all* ``Primary Stable''
and ``Secondary Stable'' code.

{}\item Is the primary responsibility of Trilinos Framework team
to test ``Stable'' code

{}\item Testing ``Stable'' code will dominate the time of the
central Trilinos testing machines

\end{itemize}


%
{}\subsection{Nightly (and any other kind of) testing of
``Experimental'' code}
%

\begin{itemize}

{}\item Is the primary responsibility of each package team

{}\item Is done (primarily) on Secondary computers associated with
each package team

{}\item Uses the testing tools and infrastructure of Trilinos to
make it easy to set up nightly cron jobs to drive testing of
``Experimental'' code (this can be made *very* easy using the
Cmake build system)

{}\item Posts results to the central Trilinos test repositories
but not in the main testing results category

\end{itemize}


%
{}\section{Testing Maintenance Issues}
%

\begin{itemize}

{}\item All ``Stable'' code should have 100\% passing tests 100\% of
the time on the primary development platforms as the norm instead of
the exception.

{}\item Achieving 100\% passing tests on secondary platforms is also a
priority but is done in a lagged secondary development loop.

{}\item A failing test on any testing platform should be addressed and
be made to pass or be disabled using the following algorithm:

  \begin{itemize}

  {}\item Fix the test in the strongest way possible

  {}\item Or, loosen the ``strength'' of test to get it pass on that
  specific platform (i.e. by loosing a platform-specific tolerance)

  {}\item Or, disable the test and submit a new item to the sprint or
  product backlog (e.g. Buzilla bug report) so that it can be
  prioritized and fixed later

  {}\item Or, remove the test and all of the associated code related
  to it

  \end{itemize}

{}\item What is significant about 100\% passing tests (i.e. 0 failing
tests)?  Mentally and in every other way, there is a massive
singularity between {}\texttt{0} failing tests an {}\texttt{X} failing
tests.  The number {}\texttt{0} failing tests == ``all passed'' and
gets everyone's attention.  When the number of failing tests goes from
{}\texttt{X} to {}\texttt{Y}, few people will pay attention to this.
This is related to the ``broken window'' phenomenon.

{}\item A 100\% passing test suite is critical for the following reasons:

  \begin{itemize}

  {}\item A 100\% passing test suite is an clear measure of the health
  of a code.

  Example: You look at the nightly test results and see 53 failing
  tests on a platform for Trilinos in what should be ``Stable'' code.
  Is the code base broken in a significant way or have these tests
  been failing for some time and someone has silently ``blessed'' them
  as minor failures?  To objectively determine the exact nature of the
  code base, you have to go back and look at test results from several
  days on various machines.  This is no way to be productive!

  {}\item Code coverage results are less meaningful when there are
  failing tests

  Example: Code coverage for package X says that there is 95\% line
  coverage.  However, there are 5 failing tests that account for a
  large percentage of the line coverage.  To really get an accurate
  code coverage count, you need to disable all of the failing tests.

  {}\item A 100\% passing test suite is an unbiased criteria for if it
  is safe to check in code

  Example: You change code in your local project, do a cvs update, and
  do a rebuild/retest before you check in and you see 5 failing tests
  in package X that depend on your package.  Were those tests already
  failing or did your change break those tests?  Without a 100\%
  passing policy, you will have to do a lot of work to determine if
  those tests pass or fail without your local changes.  This is no way
  to be productive!

  {}\item A 100\% passing test suite is an unbiased measure for if any
  code has been broken after a checkin (for example, in an
  ``asynchronous continuous integration'' build)

  Example: You just checked in code yesterday and you want to know if
  you broke any tests.  You look at the test results page and you see
  there are 8 failing tests in package X that depend on the code you
  changed.  Were those tests already failing or did you break them?
  If there is not a 100\% passing policy in strict enforcement, then
  you have to go and investigate to figure out if those test failures
  are new or if they have been present for several days.  This is no
  way to be productive!

  {}\item A 100\% passing test suite for a Trilinos package is
  important for developers of upstream packages that feed into
  downstream packages and also for developers of downstream packages
  that depend on upstream packages:

    \begin{itemize}

    {}\item A 100\% passing test suite for a Trilinos package is
    important for developers of upstream packages that feed into a
    downstream package because that provides a clear means to
    determine if changes they are making are negatively affecting
    downstream packages.  If there is even one failing test,
    developers of upstream packages will have to expend a lot of extra
    effort to determine if the failing tests are supposed to be
    failing (i.e. are old failures) or are new failures that their
    changes are causing.

    Example: You change code in your local project, do a cvs update,
    and do a rebuild/retest before you checkin and you see 5 failing
    tests in package X.  Where those tests that were already failing
    or did your change break those tests?  Without a 100\% passing
    policy, you will have to do a lot of work to determine if those
    tests pass or fail without your local changes.  This is no way to
    be productive!

    {}\item A 100\% passing test suite for a Trilinos package is
    important for developers of downstream packages in that it gives
    downstream package developers confidence that they can depend on
    and trust the upstream package developers and code in order to
    base their work on.

    Example: You do a cvs update and rebuild/retest Trilinos and you
    notice that there are now 10 failing tests in package X that your
    package Y depends on. Can you trust that package X is still
    working well enough to support your development work in package Y?
    If the developers of package X can't keep their test suites
    working, what confidence do you have that the will keep their
    library code working as well?  An environment of mistrust is very
    destructive to a development effort.

  \end{itemize}

  {}\item A 100\% passing test suite reduces the need for unnecessary
  communication within and between development teams.

  Example: There are three failing tests in package X.  You will need
  to talk with someone on the package X development team to see a) if
  they know that these tests are failing, b) if they know, then why
  they are failing and how they could be related to the code that I am
  changing.

  {}\item Running a failing test day after day is damaging to the most
  basic functions of a test suite:

    \begin{itemize}

    {}\item Chronically failing tests day after day increase the
    signal-to-noise ratio and make it very difficult to spot *new*
    regressions.

    Example: For several days in a row, the Trilinos test results
    emails say there are 5 failing test for package X.  The next day,
    it comes back with 6 failing tests.  You watch the emails and are
    expecting to see failing tests but you forget the number of
    ``expected'' failures in 5 and not 6 and this new regression goes
    unnoticed for days or weeks of time.  However, if there were
    {}\texttt{0} failing tests and then the email came back with
    {}\texttt{1} failing test, you can not help but notice that and
    can then take action.

    {}\item The presences of failing tests encourages people to check
    in more failing tests (i.e. the ``broken window'' phenomenon).

    \end{itemize}

  \end{itemize}

\end{itemize}


%
{}\section{Other Specific Areas of Possible Improvement}
%

\begin{itemize}

{}\item Reduce compilation times to speed rebuilds

  \begin{itemize}

  {}\item Take all standard C++ headers out of
  {}\texttt{PACKAGE\_ConfigDefs.hpp} and disperse them only where
  needed

  NOTE: We don't need lots of ifdefs anymore for what headers are
  present.  Instead, just include the standard $<$x$>$ C++ headers and
  the $<$cX$>$ forms of standard C headers $<$X.h$>$.

  {}\item Make greater use of forward class declarations

  {}\item Take greater advantage of the pImpl idiom for many more
  classes to:

    \begin{itemize}

    {}\item Speed builds from scratch

    {}\item Reduce recompilation

    \end{itemize}

  {}\item Use explicit instantiation for as much templated C++ code as we can:

    \begin{itemize}

    {}\item Teuchos SenseDense[Matrix,Vector] classes

    {}\item All of Thyra

    {}\item Anasazi

    {}\item Belos

    {}\item Tpetra

    {}\item ...

    \end{itemize}

  NTE : The overhead of implicit instantiation is likely causing
  segfaults on a number of compilers and is becoming a portability
  issue.

  \end{itemize}

{}\item Create different categories of tests that get built and run
for different purposes:

  \begin{itemize}

  {}\item ``Unit'' tests (i.e. TDD tests)

    \begin{itemize}

    {}\item Tests that are specific to just a single package's

    {}\item Not designed test the integration with dependent packages
    but yet will use some features from required dependent packages
    (like Teuchos)

    {}\item Build very fast

    {}\item Run very fast

    \end{itemize}

  {}\item ``Basic integration'' tests (i.e. pre-checkin tests)

    \begin{itemize}

    {}\item Includes the unit test suite (since it will test some
    integration with required packages)

    {}\item Designed to test basic integration with all required and
    optional package dependencies

    {}\item Build somewhat fast

    {}\item Run somewhat fast

    {}\item Should be run before every checkin

    {}\item Run buy the asynchronous continuous integration several

    \end{itemize}

  {}\item ``Regression'' tests (i.e. nightly tests)

    \begin{itemize}

    {}\item Includes the ``unit'' and ``basic integration'' tests

    {}\item Fully tests the functionality of all code

    {}\item Not designed primarily to address performance issues

    {}\item Takes longer to build

    {}\item Takes longer to run

    {}\item Will be run mostly at night and perhaps might also be run
    on a longer continuous integration cycle

    \end{itemize}

  {}\item ``User'' tests

    \begin{itemize}

    {}\item Similar to code that a client would write to use Trilinos code

    {}\item Does not usually include ``unit'' tests

    {}\item Provides the foundation for installation and backward
    compatibly testing

    \end{itemize}

  {}\item ``Performance'' tests

    \begin{itemize}

    {}\item Does *not* usually include the ``unit'', ``basic
    integration'', or ``regression'' tests but some test executables
    built for regression testing can also be used to drive performance
    tests.

    {}\item Designed to stress the software in terms of memory and CPU
    time on a single processor

    {}\item Takes much longer to run

    {}\item Mostly run every weekend

    \end{itemize}

  {}\item ``Scalability'' tests

    \begin{itemize}

    {}\item Does *not* usually include the other categories for tests
    but might use test executables built for other categories like
    ``performance'' tests

    {}\item Designed to run on more than one processors and to assess
    the parallel scalability of given algorithms

    {}\item Should focus on timers that time the serial work, parallel
    work, and communication overhead as separately as possible.

    \end{itemize}

  \end{itemize}

{}\item Improve the exception safety of our C++ codes (See Item 29
``Strive for exception-safe code'' in ``Effective C++ 3rd Edition'')

  \begin{itemize}

  {}\item All code should satisfy the ``basic guarantee'' (i.e. leave
  code in a valid state without leaking memory after throwing an
  exception).

    \begin{itemize}

    {}\item You can not have *any* raw calls to {}\texttt{delete} in
    your code!

    \end{itemize}

  {}\item When redos and transactioning are needed, support the
  ``strong guarantee'' (i.e. leave the code in exact same state it was
  after an exception is thrown).

  {}\item Provide the ``no thrown guarantee'' when practical or
  needed.

    \begin{itemize}

    {}\item Destructors should never be allowed to throw exceptions in
    production code

    \end{itemize}

  {}\item If code can't provide the ``basic guarantee'' as defined
  above, it should immediately abort the program.

  \end{itemize}

{}\item Improve installation testing:

  \begin{itemize}

  {}\item Configure, build and install Trilinos, then reconfigure
  Trilinos tests and examples to build against installed Trilinos
  headers

    \begin{itemize}

    {}\item In the test/example builds, only enable ``user-like''
    tests and examples (because some headers will be missing)

    {}\item Use autotools to build and install, then use Cmake to
    configure, build, and run tests/examples

    {}\item Use Cmake to build and install, then use Cmake to
    configure, build, and run tests/examples

    {}\item Cmake configuration of ``user-like'' tests and examples
    would automatically exclude source/build headers of Trilinos and
    would delete build libraries

    \end{itemize}

  \end{itemize}

\end{itemize}


%
{}\subsection{Trilinos Release Process Improvements}
%

The primary purpose to put out a ``release'' of Trilinos is to support
a client application's release to their customers.  We should not put
out releases as a way to interact with our real customers.  We should
be interacting with our customers primarily through the development
version of Trilinos to develop capabilities.

The general philosophy of the process outline below is to spread out
the work of creating a release throughout the year so that when the
branch for a release is actually created, we can do the minimal work
possible to put out the release.  The ``Stable'' code should always be
in such a close to ``releasable'' state that we should be able to
decide in a few weeks that we want to put out a release and we should
be able to get it done.

\begin{itemize}

{}\item Things to do {}\underline{before} the branch for the release
is created:

  \begin{itemize}

  {}\item Implement all functionality for the upcoming release

  {}\item Keep all documentation and examples for ``Stable'' code up
  to date after each change (i.e. definition of ``done'').  This
  includes appending the release notes and other bits of documentation

  {}\item Do all code ``clean ups''

  {}\item All non-releasable code should be confined to
  ``Experimental'' ifdefs so that it will not be included in the next
  release.

  {}\item The ``Stable'' code for each package should almost always be
  in a releasable state each and every day, regardless of when a
  release will take place.  Therefore, there should be no reason to
  branch or tag individual packages separately before the entire
  release branch is created.

  {}\item The ``Stable'' code in all of the packages to be released
  should produce clean tests on all of the test platforms well before
  the release date.  In fact, the ``Stable'' code should almost never
  show any test failures on any day of the year, regardless of when a
  release will take place.

  {}\item Perform at least one round of ports and acceptance tests
  with Trilinos Dev against all major customer platforms and
  applications a few weeks before the targeted release branch date.
  Fix any problems that are encountered.

  \end{itemize}

{}\item Things to do {}\underline{after} the branch for the release is
created:

  \begin{itemize}

  {}\item Run automated scripts to automatically strip out all
  ``Experimental'' code and tests (This removes a lot of the need for
  complex tarball logic).

  {}\item No changes are made to the branch except what are absolutely
  necessary to address serious defects.

  {}\item Do (what should be) a final round of ports and acceptance
  tests against all major customer platforms and applications.
  Resolve any new problems that have come up since the previous round
  of ports and tests conducted a few weeks prior.  (Experience with
  SIERRA + Trilinos Integration shows that very few new issues come
  up.)

  {}\item Change the version numbers inside of Trilinos. (NOTE: We
  need a more automated and uniform way of updating version numbers.)

  {}\item Create the final tag.

  {}\item Release the code.

  \end{itemize}

\end{itemize}

If the above process is done well, we should be able to do a release
of Trilinos in a week or less after the branch is created.


%
{}\section{Other Miscellaneous outstanding issues}
%

\begin{itemize}

{}\item What do we do about all of the orphaned packages in Trilinos
like Amesos, Meros, Galeri, Moertel.  Jim Willembring can not and
should not be in a position to own all of these packages.  How in the
world is Jim going to know enough about all of these packages to be
able to complete the package checklist?

{}\item Jim Willembring should not own process checklists for so many
packages ...

\end{itemize}


% ---------------------------------------------------------------------- %
% References
%
\clearpage
\bibliographystyle{plain}
\bibliography{references}
\addcontentsline{toc}{section}{References}


% ---------------------------------------------------------------------- %
% Appendices should be stand-alone for SAND reports. If there is only
% one appendix, put \setcounter{secnumdepth}{0} after \appendix
%
\appendix

%
{}\section{Problems with the current Trilinos build system and test
harness}
%

\begin{itemize}

{}\item The perl runtests script does not allow not differentiate
tests with the same test name but different argument lists.

  \begin{itemize}

  {}\item This makes it very hard to figure out exactly what the test
  is testing from looking at the results output.

  {}\item It makes it hard to track down exactly what test output to
  look at locally (i.e. you have to count 0, 1, 2, 3, ...).

  {}\item This can result in a significant under reporting of errors
  in the emails for instance (e.g. it says that only the
  {}\texttt{Rythmos\_ConvergenceTest.exe} is failing on exetazo when
  in actuality, all 5 runs with different input arguments are failing.

  \end{itemize}

{}\item The build failure detection and disabling feature is very
fragile and requires specific modifications in each and every package
to work.

NOTE:We need a system that will automatically handle that does not
require any changes to specific packages ...

\end{itemize}


\begin{SANDdistribution}[NM]
% \SANDdistCRADA	% If this report is about CRADA work
% \SANDdistPatent	% If this report has a Patent Caution or Patent Interest
% \SANDdistLDRD	% If this report is about LDRD work
% External Address Format: {num copies}{Address}
%\SANDdistExternal{}{}
%\bigskip
%% The following MUST BE between the external and internal distributions!
%\SANDdistClassified % If this report is classified
% Internal Address Format: {num copies}{Mail stop}{Name}{Org}
%\SANDdistInternal{}{}{}{}
% Mail Channel Address Format: {num copies}{Mail Channel}{Name}{Org}
%\SANDdistInternalM{}{}{}{}
%\SANDdistInternal{2}{MS 9018}{Central Technical Files}{8944}
%\SANDdistInternal{2}{MS 0899}{Technical Library}{4536}
\end{SANDdistribution}

\end{document}
