In the field of scientific computing there are many specialized programs designed for specific applications 
in areas like biology, chemistry, and physics. These applications are often very powerful and extraodinarily 
useful in their respective domains. However, many suffer from a common problem: a poor user interface. Many 
of these programs are homegrown, and the concern of the designer was not ease of use but rather 
functionality. The purpose of Optika is to address this problem and provide a simple, viable solution. Using
only a list of parameters passed to it, Optika can dynamically generate a GUI. This allows the user to specify 
parameters values in fashion that is much more intuitive than the traditional "input decks" used by many 
parameterized scientific applications. By leverageing the power of Optika, these scientific applications 
will become more accessible and thus allow their designers to reach a much wider audience while requiring 
minimal extra development effort.
