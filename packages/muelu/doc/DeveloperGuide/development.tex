\muelu makes heavy use of templating.  However, the developers have made every effort to hide templates
whenever possible to make the code easier to write, read, and maintain.

Virtually all \muelu classes are templated.  A class will begin something like the following:

\begin{verbatim}
namespace MueLu {
  template<class ScalarType,
           class LocalOrdinal, class GlobalOrdinal,
           class Node, class LocalMatOps>
  class RAPFactory : public MatrixFactory<ScalarType,LocalOrdinal,GlobalOrdinal,Node,LocalMatOps> {

#include "MueLu_UseShortNames.hpp"
\end{verbatim}
%
The file ``MueLu\_UseShortNames.hpp" contains type definitions (\verb!typedef!'s) that significantly shorten
templated class names.  For example, instead of having to write
\begin{verbatim}
MueLu::Level<ScalarType,LocalOrdinal,GlobalOrdinal,Node,LocalMatOps>
\end{verbatim}
you can simple write \verb!Level!  within the \verb!RAPFactory! class because of the type definition
\begin{verbatim}
typedef MueLu::Level<ScalarType,LocalOrdinal,GlobalOrdinal,Node,LocalMatOps> Level;
\end{verbatim}
``MueLu\_UseShortNames.hpp" itself includes the file ``Xpetra\_UseShortNames.hpp", which contains
\verb!typedef! abbreviations for \xpetra classes and for template parameters.  The template
type definitions are
%
\begin{verbatim}
typedef ScalarType    SC;
typedef LocalOrdinal  LO;
typedef GlobalOrdinal GO;
typedef Node          NO;
typedef LocalMatOps   LMO;
\end{verbatim}
