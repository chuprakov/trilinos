\subsection{Making sense of compile errors with templated code}
The Perl script STLFilt by Leor Zolman is a useful tool for making compiler errors for templated code easier to read.  You
can get the code from \url{http://www.bdsoft.com/tools/stlfilt.html}.
Put the files \verb!gSTLFilt.pl! and \verb!gfilt! in a directory that is on your search path.
Whenever you encounter a compile error, duplicate the compile invocation line, replace the compiler with \verb!gfilt!, and recompile.
If you are using the MPI compiler wrappers, you'll need to add the MPI include directory along with
\verb!gfilt!.

Section \ref{sec:common errors} lists some C++ compilation errors that the \muelu developers have seen, along
with explanations and solutions.

\subsection{Using tags to navigate the \muelu source code}
You can use a tag file to jump from the name of a class/method/function
to its definition from within vim.   Assume that the \muelu source is in \verb!~/Codes/MueLu!.  First generate a tag file:
\begin{verbatim}
  cd ~/Codes/MueLu
  ctags -R --tag-relative=no ${PWD}/src
\end{verbatim}
The file ``tags" should now exist in \verb!~/Codes/MueLu!.  Second, edit the file \verb!~/.vimrc!, and add the
line:
\begin{verbatim}
autocmd BufEnter ~/Codes/MueLu/* :setlocal tags=~/Codes/MueLu/tags
\end{verbatim}
This line causes the tags files to become active whenever you edit a file in \verb!~/Codes/MueLu!.  To test this, edit
the file \verb!MueLu/test/unit_tests/Level.cpp!, and move the cursor over the type \verb!Level!.  Type ``control ]",
and you will jump to the class definition of \verb!MueLu::Level!.  To jump back, type ``control t".
In vim, type \verb!:help tags! for extensive help on tags.
