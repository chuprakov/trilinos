\section{Example: A Conjugate Gradients Solver}

The best way to introduce the TSF objects is through example, so let's dive in and
write a solver. We'll write a simple program to solve $Ax=b$ using unpreconditioned
CG which we code in-line. 


\subsection{Step-by-step explanation}

We start with a step-by-step walkthrough of the CG solver example.
When finished, there will be a summary and then the complete CG code will be
listed for reference.

\subsubsection{Boilerplate}

A dull but essential first step is to show 
the boilerplate C++ common to any TSF main program. 
\begin{verbatim}
#include "TSF.h"

int main(int argc, void** argv)
{
  try
    {
      TSF::init(argc, argv);

      /*
       * code body goes here
       */
    }
  catch(exception& e)
    {
       TSF::handleException(__FILE__, e);
    }
  TSF::finalize();
}
\end{verbatim}
The body of the code -- everything else we discuss here -- goes in place 
of the comment {\tt code body goes here}. 

\subsubsection{Specifying the vector type}

TSF is designed to work with arbitrary vector representations. 

\begin{verbatim}

\end{verbatim}







