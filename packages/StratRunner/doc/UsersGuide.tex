\documentclass{article}
\title{How to use StratRunner}
\author{Kurtis Nusbaum}
\date{February 27th 2009}
\begin{document}
\maketitle

\section{Introduction}
StratRunner is a Graphical Interface for the Stratimikos package in the Trilinos Project. It is inteded to assist users in defining solvers
and applying those solvers to Matricies. The hope is that users who are unfamiliar with the Trilinos Project or lack the technical ability
to write their own C++ code will be able to access some of the power inside of Trilinos. It is worth noting that StratRunner is a Graphical User
Interface. While an attempt was made to minimize the compromising of controll on the users part, some simplifications have been made. I feel that even
power users of the Stratimikos package will find this GUI usefull.

\section{Creating a New Solver from Scratch}

When you first start StratRunner you are prompted with a choice, you can either load an existing solver or start a new one. By clicking the new solver
button, the user is present with the primary StratRunner interface. Here the user can modify any and all of the parameters available in the Stratimikos
package. While at some points there may seem to be an overwhelming amount of parameters, know this: all parameters in StratRunner are set to appropriate 
defaults. If you don't know what a certain parameter does, then just leave it alone because it is already set to its default setting.

To change a particular parameter, simply right click on it's name, and select "Change Parameter" form the context menu that pops up. A list of
appropriate choices or an appropriate input type will be presented to you in the form of a dialog window. If the parameter is a boolean the context
menu will simply have either the option to set it true, or set it false.

To see the all of the parameters included withing a particular Parameter List, simply click the "+" sign next to the name of the Parameter List. To
minimize the Parameters displayed within a Parameter List, simply clikc the "-" sign next to the name of the Parameter List.

\section{Loading a Solver}

When you first start StratRunner you may load an exisiting solver that you have previously saved, or one that you have written yourself by hand (note
that you must have written this solver as an .xml file). To load the solver, click the load solver button and you will be prompted with a dialog box.
Navigate to the folder containing your solver and select it. Your solver will then be loaded into StratRunner and all of the Parameters will now be
set to correspond with those specified in the file you just loaded.

You may also load a solver from the primary interface. Click on the file menu and select the "Load" option.

\section{Running a Solver}

In order to run a solver you must first save the one you are currently working on. This can be done by going to the "File" menu and selecting
either "Save" or "Save As". Note that if you have just loaded a file, you do not need to save it.

Once you are ready to run your solver, click on the "Run" menu and select the "Open Run Window" option. A new window will open up. This is called the 
"Run Window". From it you may apply the current solver that you are working on to any matrix file. Click the "Open Matrix" button to select a matrix
file to use. The matrix must be in Market Matrix file format (.mtx).

Once you have selected the matrix file, you may click the "Run" button. This will use the solver you have created to solve the matrix that you have
specified. All the results will be displayed in the text area in the Run Window. You may save these results to a file by clicking the "Save Output"
button in the lower right hand corner of the Run Window.

\section{Contact}

If you are having any problems with StratRunner, please contact Kurtis Nusbaum at klnusbaum@csbsju.edu.

\section{Acknoledgements}

I would like to thank Dr. Mike Heroux for giving me the wonderful opportunity to do reasearch for him. It has been a truely enjoyable experience. I 
would like to thank Jonathan Hu and Christopher Baker for there help in determining options for certain parameters. Lastly I would like to thank all
my Family, Friends, and specifically my Mother and Father. You have all been so amazing and there are not words to express my gratitued towards you all.
My love for you is infinite.
\end{document}
