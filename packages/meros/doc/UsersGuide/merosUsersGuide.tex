% Meros User's Guide
%
% Formatted for Sandia SAND Report
\typeout{Using SANDmath LaTeX Package: 6/12/2007, VEH}

\documentclass[oneeqnum,onefignum,onetabnum,10pt]{SANDreport}
\usepackage{amsmath}
\usepackage{amsfonts}
\usepackage{amssymb}
\usepackage{epsfig}
\usepackage{ifthen}

\usepackage{mathptmx}

\usepackage{theorem}
\usepackage{latexsym}
\usepackage{thp}
\usepackage{subfigure}
\def\vdown{\\[4pt]}

\setlength{\topmargin}{0in}
\setlength{\oddsidemargin}{.10in}
\setlength{\evensidemargin}{.10in}
\setlength{\headheight}{1pt}
\setlength{\headsep}{10pt}
\setlength{\topskip}{1pt}
\setlength{\textheight}{8.5in}
\setlength{\textwidth}{6.25in}

\definecolor{AztecOOBlue}{rgb}{0,0,0}
\def\TrilinosTM{{\color{AztecOOBlue} \bf
      \textsf{Trilinos}}\textsuperscript{\tiny{\texttrademark}}}
\def\MerosTM{{\color{AztecOOBlue} \bf
      \textsf{Meros}}\textsuperscript{\tiny{\texttrademark}}}
\newcommand{\InlineCommand}[1]{
  {\hspace{0.01 in}} {\tt #1} {\hspace{0.01 in}}}

\input{epsf}

\title{DRAFT: Meros User's Guide\footnote{For \MerosTM{} Version 2.0 in
\TrilinosTM{} Release 8.0}}

%\thanks{This work was partially supported
%by the DOE Office of Science MICS Program and by the ASC Program at
%Sandia National Laboratories.  Sandia is a multiprogram laboratory
%operated by Sandia Corporation, a Lockheed Martin Company, for the
%United States Department of Energy's National Nuclear Security
%Administration under contract DE-AC04-94AL85000.}}

\author{
Victoria E. Howle\thanks{Sandia National Laboratories, PO Box 969, MS 9159
    Livermore, CA 94551, {\tt vehowle}\protect@{\tt sandia.gov}.}
  \and
  Robert Shuttleworth\thanks{Applied Mathematics and Scientific Computing Program and Center for Scientific
Computation and Mathematical Modeling,
University of Maryland, College Park, MD 20742. {\tt rshuttle}\protect@{\tt math.umd.edu}}
  \and
  Ray Tuminaro\thanks{Sandia National Laboratories, PO Box 969, MS 9159,
    Livermore, CA 94551, {\tt rstumin}\protect@{\tt sandia.gov}.}
}

\date{}

\SANDnum{SAND2007-xxxx}
\SANDprintDate{August 2007}
\SANDauthor{Victoria E. Howle, Robert Shuttleworth, Ray Tuminaro}

\begin{document}

\maketitle
\begin{abstract}
meros abstract
\end{abstract}




\bibliographystyle{siam}
%\pagestyle{plain}

\SANDmain

\clearpage
\tableofcontents
%\listoffigures
%\listoftables

%\input{intro}
\section{Notation}
\begin{verbatim} 
% a_really_long_command 
\end{verbatim}
The character \verb!%! indicates any LINUX or UNIX shell prompt.
Function names are shown as {\sf LSCSchurFactory}.  Names of packages
or libraries as reported in small caps, as {\sc Epetra}. Mathematical
entities are shown in italics.


\section{Introduction}
Meros is a segregated preconditioning package. Provides scalable block
preconditioning for problems that coupled simultaneous solution
variables such as Navier-Strokes problems. 


Adding a citation to test bib\cite{ElmanSilvesterWathen.book}.

\section{Block Methods}
\subsection{Pressure Convection-Diffusion (PCD)}
Factory for building pressure convection-diffusion style block
preconditioner. This class of preconditioners were originally proposed
by Kay, Loghin, and Wathen (ref) and Silvester, Elman, Kay, and Wathen
(ref).

Meros 1.0 currently implements the PCD preconditioner, a.k.a. Fp
preconditioner. 

The LDU factors of a saddle point system are given as follows:

\begin{equation}
  \left[ \begin{array}{cc} A & B^T \\ B & C \end{array} \right]
     = \left[ \begin{array}{cc} I & \\ BF^{-1} & I \end{array} \right]
       \left[ \begin{array}{cc} F & \\  & -S \end{array} \right]
       \left[ \begin{array}{cc} I & F^{-1} B^T  \\  & I \end{array} \right],
\end{equation}

where $S$ is the Schur complement $S = B F^{-1} B^T - C$.  A
pressure convection-diffusion style preconditioner is then given by

\begin{equation}
  P^{-1} =
       \left[ \begin{array}{cc} F & B^T \\ & -\tilde S \end{array} \right]^{-1}
       = 
       \left[ \begin{array}{cc} F^{-1} &  \\  & I \end{array} \right]
       \left[ \begin{array}{cc} I & -B^T \\  & I \end{array} \right]
       \left[ \begin{array}{cc} I &  \\  & -\tilde S^{-1} \end{array} \right]
\end{equation}
where for $\tilde S$ is an approximation to the Schur complement S.

To apply the above preconditioner, we need a linear solver on the
(0,0) block and an approximation to the inverse of the Schur
complement.

To build a concrete preconditioner object, we will also need a 2x2
block Thyra matrix or the 4 separate blocks as either Thyra or Epetra
matrices.  If Thyra, assumes each block is a Thyra EpetraMatrix.





\subsection{Least Squares Commutator (LSC)}
Factory for building least squares commutator style block
preconditioner.  

Note that the LSC preconditioner assumes that we are using
a stable discretization an a uniform mesh.

The LDU factors of a saddle point system are given as follows:

\begin{equation}
  \left[ \begin{array}{cc} A & B^T \\ B & C \end{array} \right]
     = \left[ \begin{array}{cc} I & \\ BF^{-1} & I \end{array} \right]
       \left[ \begin{array}{cc} F & \\  & -S \end{array} \right]
       \left[ \begin{array}{cc} I & F^{-1} B^T  \\  & I \end{array} \right],
\end{equation}
where $S$ is the Schur complement $S = B F^{-1} B^T - C$.
A pressure convection-diffusion style preconditioner is then given by
     
\begin{equation}
     P^{-1} =
       \left[ \begin{array}{cc} F & B^T \\ & -\tilde S \end{array} \right]^{-1}
       = 
       \left[ \begin{array}{cc} F^{-1} &  \\  & I \end{array} \right]
       \left[ \begin{array}{cc} I & -B^T \\  & I \end{array} \right]
       \left[ \begin{array}{cc} I &  \\  & -\tilde S^{-1} \end{array} \right]
\end{equation}
where for $\tilde S$ is an approximation to the Schur complement S.

To apply the above preconditioner, we need a linear solver on the
(0,0) block and an approximation to the inverse of the Schur
complement.

To build a concrete preconditioner object, we will also need a 2x2
block Thyra matrix or the 4 separate blocks as either Thyra or Epetra
matrices.  If Thyra, assumes each block is a Thyra EpetraMatrix.




\subsection{SIMPLE}

\section{Examples}

\appendix

\section{Meros \InlineCommand{configure} Options}


\bibliography{merosrefs}
    
\include{SANDdistribution}

\end{document}
