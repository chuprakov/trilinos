\documentclass[10pt]{article}
%\documentclass[10pt,twocolumn]{article}

\pagestyle{empty}

\usepackage[margin=1.0in]{geometry}
\usepackage{paralist}
\usepackage{verbatim}
\newtheorem{listing}{Listing}

\usepackage{natbib}
\setlength{\bibsep}{0.0cm}

\usepackage{hyperref}

%
% Command to print 1/2 in math mode real nice
%
\newcommand{\myonehalf}{{}^1 \!\!  /  \! {}_2}

%
% Command to print over/under left aligned in math mode
%
\newcommand{\myoverunderleft}[2]{ \begin{array}{l} #1 \\ \scriptstyle #2 \end{array} }

%
% Command to number equations 1.a, 1.b etc.
%
\newcounter{saveeqn}
\newcommand{\alpheqn}{\setcounter{saveeqn}{\value{equation}}
\stepcounter{saveeqn}\setcounter{equation}{0}
\renewcommand{\theequation}
	{\mbox{\arabic{saveeqn}.\alph{equation}}}}
\newcommand{\reseteqn}{\setcounter{equation}{\value{saveeqn}}
\renewcommand{\theequation}{\arabic{equation}}}

%
% Shorthand macros for setting up a matrix or vector
%
\newcommand{\bmat}[1]{\left[ \begin{array}{#1}}
\newcommand{\emat}{\end{array} \right]}

%
% Command for a good looking \Re in math enviornment
%
\newcommand{\RE}{\mbox{\textbf{I}}\hspace{-0.6ex}\mbox{\textbf{R}}}
\newcommand{\CPLX}{\mbox{\textbf{C}}\hspace{-1.1ex}\mbox{l}\hspace{0.4ex}}

%
% Commands for Jacobians
%
\newcommand{\Jac}[2]{\displaystyle{\frac{\partial #1}{\partial #2}}}
\newcommand{\jac}[2]{\partial #1 / \partial #2}

%
% Commands for Hessians
%
\newcommand{\Hess}[2]{\displaystyle{\frac{\partial^2 #1}{\partial #2^2}}}
\newcommand{\hess}[2]{\partial^2 #1 / \partial #2^2}
\newcommand{\HessTwo}[3]{\displaystyle{\frac{\partial^2 #1}{\partial #2 \partial #3}}}
\newcommand{\hessTwo}[3]{\partial^2 #1 / (\partial #2 \partial #3)}



%\newcommand{\Hess2}[3]{\displaystyle{\frac{\partial^2 #1}{\partial #2 \partial #3}}}
%\newcommand{\myHess2}{\frac{a}{b}}
%\newcommand{\hess2}[3]{\partial^2 #1 / (\partial #2 \partial #3)}

%
% Shorthand macros for setting up a single tab indent
%
\newcommand{\bifthen}{\begin{tabbing} xxxx\=xxxx\=xxxx\=xxxx\=xxxx\=xxxx\= \kill}
\newcommand{\eifthen}{\end{tabbing}}

%
% Shorthand for inserting four spaces
%
\newcommand{\tb}{\hspace{4ex}}

%
% Commands for beginning and ending single spacing
%
\newcommand{\bsinglespace}{\renewcommand{\baselinestretch}{1.2}\small\normalsize}
\newcommand{\esinglespace}{}



\raggedright

\title{
TriBITS Overview \\[1.5ex]
\large The Tribal Build, Integrate, and Test System
}

\author{Roscoe A. Bartlett, Oak Ridge National Laboratory}

\date{??/??/???}

\begin{document}

\maketitle

\begin{abstract}

The Tribal Build, Integrate, and Test System (TriBITS) is a framework built on top of the open-source CMake set of tools designed to handle large software development projects involving multiple independent development teams and multiple source repositories.  TriBITS defines a complete agile software development, testing, and deployment system supporting processes consistent with modern agile software development best practicies.  This paper provides an overview of TriBITS that describes its roots, what problems and types of projects it is designed to address, and then describes the major archetectural design and features of the system.  This presentation should provide the reader with a fairly good idea what TriBITS is, that it is not, what the advantages and disadvantages are, and what projects using TriBITS look like.

\end{abstract}

\tableofcontents

%
\section{Introduction}
%

Developing, testing, and deploying complex computational software involving multiple compiled languages developed by multiple distrubuted teams of developers that must be deployed on many different platforms on an aggressive schedule is a daunting endeavor.  While the modern software engineering (SE) community has made great strides in developing principles, processes and best practicies to manage such projects (e.g.\ agile methods \cite{CodeComplete2nd04, AgileSoftwareDevelopment, ContinuousIntegration07, XP2, TDD}), it takes non-trivial tools to effectively implement such processes.  In addition, just the mechanics of configuring, building, and installing complex compiled multi-language software on a variety of platforms is very challanging.  Today's computational software must be able to be run on platforms ranging from basic Linux workstations and Microsoft Windows machines to the largest bleeding-edge massively parallel super computers.

TriBITS is an attempt to create an organized framework built on the Kitware CMake tools to address many of these challanges across a potentially large number of semi-independent development efforts while still allowing for seamless integration and deployment for large stacks of related software.  On the low end, TriBITS can be used to quickly develop a small independent software product with all the bells and whistles of agile software development including pre- and post-test continuous integration \cite{XP2, ContinuousIntegration07}.  On the high end, TriBITS can be used as a meta-build system to integrate several smaller semi-independent TriBITS-enabled software projects.  In addition, TriBITS directly supports the customized modern Lean/Agile research to production lifecycle processes defined in \cite{TribitsLifecycleModel, TribitsLifecycleModel_eScience2012}.

The scope of TriBITS and the material in this paper is the development and deployment of software written in multi-language compiled lanuages, primarily C, C++ (93, 03, 11), and Fortran (77, 90, 95, 2003, 2008) which use MPI and various local threading approaches to achieve parallel computation.  Creating an effective development and deployment system for these types of software projects and languages is difficult.

ToDo:
* Discuss the history of TriBITS
* Mention the CASL project and how it is using TriBITS
* Mention the other projects using TriBITS

%
\subsection{Why CMake?}
\label{sec:why_cmake}
%

Many different build and test tools have been created and are avaiable in the open-source community.  Many computational projects just use raw Make or GNU Make and devise their own add on scripts to drive configuration, building, and testing.  For simple projects that don't need to be very portable and only need to run on Linux, writing raw makefiles is attractive.  However, for example, raw makefiles will not automatically rebuild objects files, libraries and executables when C and C++ header-files change.  Another popular set of tools used in the computational community are the so-called GNU autotools which are comprised of autoconf, automake, and related programs.  Using autotools over raw (GNU) Make offers several advantages but these tools were never designed to manage the development and deployment of large complex software projects.  We have worked with projects that have extensive experience using raw makefiles, autotools, and other home-grown tools based on these (e.g.\ \cite{Trilinos}).  There are other build and test systems as well but we will not mention those here (this paper is not meant to be an exhaustive literature review of build and test tools).  We only mention raw makefiles and GNU autotools because in our observation, these are the most popular approaches currently in use on these types of software projects.

Then there was CMake.  What is CMake and what does it offer for these types of complex mixed-lanauge software projects?  When CMake is installed (which just requires a basic C++ compiler and little else to get the key functionality), it actually provides the following tools:

\begin{compactitem}
\item\textit{\textbf{CMake}}: Portable configuration and build manager that includes a complete scripting language for configuring and building software libraries and executables that leverages native build systems and IDEs (e.g. cmake can generate project files for Make, Ninja, MS Visual Studio, Eclipse, and XCode).
\item\textit{\textbf{CTest}}: Executable to handle running tests in the shell and reporting results to CDash.
\item\textit{\textbf{CPack}}: Cross-platform software packaging tool, with installer support for all systems currently supported by CMake.
\end{compactitem}

In addition, a complementary build/test dashboard implementation that is well integrated with CMake and CTest is provided by Kitare that can be installed separately:

\begin{compactitem}
\item\textit{\textbf{CDash}}: build and test reporting dashboard built on PHP, CSS, XSL, MySQL, and Apache HTTPD
\end{compactitem}

When the Trilinos project was first considering adopting CMake, a detailed evaluation was performed \cite{TrilinosCMakeEvaluation08} comparing its usage to the existing customized Trilinos autotools system which included a home-grown perl-based test execution harness and dashboard.  As found as part of that evaluation and the through years of usage by the Trilinos and related projects, we have found that the major (but by no means not all of the) advantages/capabilities of CMake and CTest over autotools and the existing Trilinos perl-based test harness were:

\begin{compactitem}
\item Simplified build system and easier build and test maintenance.
\item Improved mechanism for extending capabilities using simple, fast CMake scripting lanaguage (as compared to M4 in autotools).
\item Built-in support for all major C, C++, and Fortran compilers.
\item Automatactic built-in full dependency tracking of every kind possible on all platforms (i.e. header to object, object to library, library to executable, and any build system changes).
\item Faster configure times (e.g.\ from minutes with autotools to seconds with CMake).
\item Built-in support for shared libraries on a variety of platforms and complers.
\item Built-in support for MS Windows (i.e. generates Visual Studio projects, Windows installers, etc.).
\item Support for cross-compiling (i.e.\ build on compile nodes but run on a compute nodes).
\item Built-in automatic depencency tracking for Fortran 90+ module files and object files to allow parallel Fortran compilation and linking.
\item Built-in support for portable determination of C/C++/Fortran mixed-langauge bindings.
\item Parallel running and scheduling of tests and test time-outs (i.e. run 1000 test executabes each using different numbers of cores effectively on 16 processes keeping all 16 processes busy).
\item Memory testing with Valgrind and purify that is backed up by Kitware and a large development community.
\item  Line coverage testing that is backed up by Kitware and a large development community.
\item Better integration between the test system and the build system (e.g. natural and flexible specification based on what builds, platform consideraitons, etc.).
\item Leverages open-source tools maintained and used by a large community.
\end{compactitem}

While there are other build and test systems with significant advantages of raw makefiles and autotools that have been developed, the CMake set of tools is more widely used and its usage is growing in computational science projects.  There is safety is numbers and the decision to adopt CMake for Trilinos has generally been considered to be a very positive development in the project and has allowed for the scalable growth of Trilinos in every way.

A drawback of CMake over autotools is that every client that needs to build and install software on a given system to to have the minimum a compable version of CMake installed first.  (Raw makefiles, autotools, and CMake all require a recent version of Make installed on a Unix/Linux system.)  Kitwware provides free downloads of binary installers for CMake for a variety of platforms and CMake is easy to configure and install from source with just a C++ compiler.  In general, we have found CMake to be much easier to configure and install on a given platform than to subsequently configure, build, and install any project that uses cmake.  Therefore, we have found the additional overhead of having to install CMake is in the noise compared to installing the target software.

%
\subsection{Why TriBITS?}
%

So if CMake so many advantages and featuers over autotools and may other alternatives, then why not just use raw CMake?  While the built-in features that one gets with just the strightforward raw usage of CMake (as mentioned in Section~\ref{sec:why_cmake}) are signficiant, there are several problems and shortcomings with directly using only raw CMake commands in a large project.  TriBITS is designed to address these problems and shortcomings and has been demonstrated to do so successfully (for at least the projects where TriBITS has been used).

At is most basic, TriBITS provides a framework for CMake-based projects that leverages all the advantages/features of raw CMake, CTest, CPack, CDash, but in addition provides the following (in relative order of significance):

\begin{compactitem}
\item Provides a set of wrapper CMake functions and macros to reduce boiler plate CMake code and enforces consistancy across the entire project.
\item Provides a subproject dependency and namespacing architecture (i.e. packages with required and optional dependencies and namespaced CMake target and test names).
\item Provides additional tools to enable better and more efficient agile software development and deployment processes.
\item Adds basic additional functionality missing in raw CMake.
\item Changes default CMake behavior when neccesary and beneficial to the project.
\end{compactitem}

To see the potential problems and shortcoming of just using raw CMake, consider the simple CMakeLists.txt file shown in Listing~\ref{listing:RawHelloWorld} that builds and installs a library from a set of C++ sources and then creates a user exectuable and a set of unit test exectuables and tests.

\begin{listing}: Simple example hello world raw CMakeLists.txt file
\label{listing:RawHelloWorld}
{\small
\verbatiminput{../examples/RawHelloWorld/hello_world/CMakeLists.txt}
}
\end{listing}

Looking at the raw CMake code in Listing~\ref{listing:RawHelloWorld}, it is pretty clear what it is doing, even if one is not familar with CMake.  A large project will have hundreds if not thousands of CMakeLists.txt files like this that define various libraries, executables, and tests (hopefully lots of tests if it is good software).  So what is wrong with this raw CMake code?  If one appraoches this from a basic software coding perspective (see \cite{CodeComplete2nd04}), there are several problems with these raw CMakeLists.txt files.  First, there is a lot of duplication of the same information.  For example, it duplicates the library name \ttt{hello\_world\_lib} four times just in this very short file.  With duplication comes the opportunity to mispell names, forgetting to update the name when it changes under maintenance, etc.  Also, when you define a test and set a regex to check the output, you have to list the test name at least twice.  Again, more duplication.  Another problem is that while the library name \ttt{hello\_world\_lib} and the executable name \ttt{hello\_world} are likely pretty well namespaced, the target names for the tests \ttt{test} and \ttt{unit\_tests} are not.  If another directory uses the same test names, they will clash which is an error in CMake.  While CMake varibles are namespace based on \ttt{ADD\_SUBDIRECTORY()} and \ttt{FUNCTION} scopes, the targets and tests that are created are not, they must be globally unique!  The problem is that CMake provides no namespacing facility for these types of entities.  It is up the the developers of a large project to come up with their own namespacing scheme for global targets, libraries, and executables.

Another set of problems with raw CMake code like in Listing~\ref{listing:RawHelloWorld} relates to needing to set consistent policies and change global behaviors across all CMake code inside of large projects.  For example, what if an executable-only installation mode needs to be added to the project where only the created execuables and supporting scripts are installed and not the header files and libraries?  How would that be accomplished with raw CMakeLists.txt files like this?  In order to support that, the raw CMakeLists.txt files would need to be modifed to put in if statements around the header and library \ttt{INSTALL()} commands based on a CMake varible set by the user.  However, if shared libraries are used, one must install the shared libraries as well but not so when static libraries are used.  Adding this feature to a project using raw CMake commands would mean updating hundreds if not thousands of CMakeLists.txt files in a large project.  Like this there are several other project-wide polcies and features that one would like the ability to add without having to manually modify hundreds if not thousands of CMakeLists.txt files.

What the above example raw CMakeList.txt file in Listing~\ref{listing:RawHelloWorld} and the above discussion tries to demonstrate is that for many large projects using raw CMake commands has a number of disadvantages.  That is where TriBITS comes in.  To see how TriBITS helps, the equivalent TriBITS CMakeLists.txt file for this example project is given in Listing~\ref{listing:TribitsHelloWorld}.

\begin{listing}: Simple example hello world TriBITS CMakeLists.txt file
\label{listing:TribitsHelloWorld}
{\small
\verbatiminput{../examples/TribitsHelloWorld/hello_world/CMakeLists.txt}
}
\end{listing}

Comparing Listing~\ref{listing:TribitsHelloWorld} and Listing~\ref{listing:RawHelloWorld}, one can immediately see the elimination of a lot of boiler-plate CMake code and a reduction in duplication resulting in far fewer CMake commands and lines of CMake code.  The showen macros \ttt{TRIBITS\_ADD\_LIBRARY()}, \ttt{TRIBITS\_ADD\_EXECUTABLE()}, \ttt{TRIBITS\_ADD\_TEST()}, \ttt{TRIBITS\_ADD\_EXECUTABLE\_AND\_TEST()} (and others that are defined by TriBITS not shown in this example) eliminate a lot of boiler-plate CMake code and reduce duplication while covering 99\% of the use cases for defining libraries, executbles, and tests for the large projects where TriBITS has been applied.  To briefly describe how this works, TriBITS defines packages as a unit of encapsulation and namespacing (see Section ??? for detials).  In the simple example in Listing~\ref{listing:TribitsHelloWorld}, an entire TriBITS package is defined in a single CMakeLists.txt file.  Within a TriBITS package, all executables are assumed to require the capabilities of the libraries in the given package (in this case, the HelloWorld TriBITS package).  Therefore, there is no need to continually manually call \ttt{TARGET\_LINK\_LIBRARY()} for every executable in a package (it is called automatically).  In addition, by default, the macros \ttt{TRIBITS\_ADD\_EXECUTABLE()}, \ttt{TRIBITS\_ADD\_TEST()} namespace the executable targets and the test names by the package name, so names pass into these macros only need to be unique inside of a TriBITS package, not across other TriBITS packages developed by other individuals.  (Note that the names of executables can have the package name prefix removed by passing in NOEXEPREFIX.)  By default, headers and libraries are automatically installed in \ttt{<install>/include} and \ttt{<install>/lib} which is by far more common use case for these types of projects.  More details about these macros and what they doe are given later in Section ???.

At the same time, these TriBITS macros provide a hook for adding additional features in a consistent way across a very large project (and every project, repository, and package that uses TriBITS; more on that in Section ???).  For example, adding a new policy to skip installing header files and libaries when only executables should be installed is implemented within these macros and is then applied to all packages automatically wihtout having to change a single line of existing CMakeLists.txt files.  Other features that have been implemented in TriBITS inside of these basic macros include:

\begin{compactitem}
\item Performing backward compatibility testing (see Section ???).
\item Automatically calling MPI-enabled tests using the correct MPI run command and syntax.
\item ???
\end{compactitem}

Without having a set of basic wrapper macros around raw CMake commands, it is impossible to consistently add these types of features or enforce policies cross a large CMake project.

Now that basic background on CMake and TriBITS has been presented and a basic value statement for TriBITS has been stated, the remainder of this paper seeks to provide more details about how TriBITS projects are layed out, and what the various features TriBITS offers.

%
\section{Overview of the TirBITS system}
%

%
\subsection{TriBITS Requirements, Architecture and Design Principles}
%

Requirements for TriBITS:
* 
* 100\% automatic correct handling of dependencies for C and C++ header files to object files, libraries, and executables.
* Direct support for all major platforms including Linux, MS Windows,  all major super computers and every significant system in between.
* 

%
\subsection{TriBITS Package Dependency Handling}
%

* Packages, subpackages, and TPLs
* Enabling upstream and downstream packages and tests

%
\subsection{Example TriBITS Project}
%

ToDo:
* Use the TribitsExampleProject to demonstrate the structure.


%
\subsection{TriBITS Project Development Workflow}
%


%
\subsection{TriBITS Testing and Integration}
%

* Show diagram of differnet test categories
* Describe checkin-test.py


%
\subsection{TriBITS Support for Multiple Source Respositories and Distributed Development Groups}
%


%
\section{Summary}
%

ToDo: Fill in!

There are, however, a few disadvantages to using TriBITS over raw CMake that must be calked.  First, using these encapsulating scripts, one can almost forget they are using CMake and can even avoid leaning CMake basics.  This becomes a problem one needs to do something slightly outside of the normal 99\% use case coverged by basic TriBITS usage.




\bibliographystyle{plain}
\bibliography{references}

\end{document}
