\section{Introduction}
In the world of scientific computing there is a problem: most software developers are far more concerned
with the functionality of their software rather than their user interface. This is understandable given
the limited time and pressures of scientific computing environments. And in 
cases where there are only a few users of a piece of software this type of development is tolerable. However, when a piece of
software starts to be used by a wider audience, poor user interface design issues come to the forefront and 
can greatly hinder further adoption of a particular piece of software. Optika\footnote{For more information on Optika, please see
its documentation~\cite{OptikaPackage}} is an attempt to solve this
problem in a generic fashion for parameterized scientific applications.

Since developers of scientific applications don't really care about user interfaces, Optika needs
to provide a minimal amount of hurdles for developers. Also, the end result needs to be an intuitive
user interface that can be easily navigated and utilized regardless of the underlying computations being done.

The purpose of this paper is to discuss the development of the Optika package. In doing so we hope to
demonstrate how Optika solved some of the issues associated with developing a generic user interface
for scientific applications and provide justification for why we chose particular solutions. We will
proceed to discuss Optika development in a semi-chronological fashion.

